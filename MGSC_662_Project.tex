\documentclass[12pt]{article}

\usepackage{graphicx} % For including graphics
\usepackage{titling} % For customizing the title section
\usepackage{color}
\usepackage[margin=1in]{geometry} % For page margins
\usepackage{graphicx} % For including graphics
\usepackage{hyperref} % For hyperlinks in the table of contents
\usepackage{graphicx} % For including graphics
\usepackage{hyperref} % For hyperlinks in the table of contents
\usepackage{amsmath} % For mathematical formulas
\usepackage{listings} % For code listings
\usepackage{xcolor} % For coloring code
\usepackage{amsfonts} % For mathematical fonts
\usepackage{amssymb} % For mathematical symbols
\usepackage[utf8]{inputenc}
\usepackage[margin=1in]{geometry}
\usepackage{bm}


% Your title here
\title{Disaster Route Planning}
\author{} % Empty author to remove it from title area
\date{} % Empty date to remove it from title area

% Begin Document
\begin{document}

% Title Page
\begin{titlepage}
    \centering
    \vspace*{1 cm}
    \textbf{\Large Disaster Route Planning in Montreal}\\[2 cm] % Title of your document
    \includegraphics[scale=0.5]{mcgill_logo.png}\\[1 cm] % Replace with your college logo
    \textbf{Group 3:}\\[0.5 cm]
    Arnav G. -- 260658711\\
    Lakshya A. -- 261149449\\
    Michael M. -- 261060598\\
    Nandani Y. -- 261137002\\
    Om S. -- 261112933\\[1 cm]
    \textbf{MGSC 662 Decision Analytics}\\[0.5 cm]
    Professor Javad Nasiry\\[1 cm]
    \textbf{Desautels Faculty of Management}\\
    \textbf{McGill University}\\[2 cm]
    \textcolor{red}{\hrulefill}\\[0.5 cm]
\end{titlepage}


% Table of Contents
\tableofcontents
\newpage


% Introduction
\section{Introduction}
Disasters, both natural and man-made, pose a significant challenge in urban areas, necessitating effective planning and rapid response strategies. This project introduces an advanced optimization system focused on enhancing disaster response through intelligent route planning for bus-based evacuations. In the face of emergencies, traditional transit systems often become disrupted, emphasizing the need for efficient, adaptable evacuation operations.

The significance of disaster planning cannot be understated, especially in densely populated urban environments. Effective disaster response requires not only immediate action but also strategic foresight and planning. In many cities, the existing infrastructure, while robust under normal circumstances, may not be equipped to handle the sudden and intense demands of a disaster scenario. This creates a critical need for systems that are capable of anticipating and adapting to the evolving nature of emergencies.

In this study, the dynamic and unpredictable challenges of disaster response are addressed with a focus on optimizing transportation logistics. Developing a system that can adapt in real-time to the changing requirements of a disaster situation is crucial. Montreal, with its dense population and intricate transit network, exemplifies an urban landscape where efficient disaster response is vital. The proposed system in this project aims to utilize the city's existing transit infrastructure effectively while introducing flexible measures to ensure rapid and safe evacuation during times of crisis.

Emergency preparedness, particularly in urban settings, involves a multifaceted approach. It requires integrating technological solutions with strategic planning, ensuring that responses are not only swift but also comprehensive and considerate of various potential scenarios. By optimizing evacuation routes and leveraging existing transit networks, urban areas can enhance their resilience and preparedness for various disaster situations.

This project's exploration into route optimization for disaster evacuations in Montreal offers insights into the broader field of urban disaster response. It presents a model that can be instrumental in improving evacuation strategies, ultimately contributing to the safety and well-being of urban populations in times of crisis.

\section{Problem Description}

\subsection{Context}

In Montreal, the optimization of evacuation routes takes on added complexity due to the potential for diverse disaster scenarios. The city's layout presents a unique set of challenges, where traditional transit solutions may fall short during emergencies. This project aims to address these complexities by developing a routing model that is not only efficient in normal conditions but also highly adaptable in emergency situations. The goal is to ensure that evacuation planning is robust, responsive, and capable of handling the sudden shifts in transit dynamics that disasters often bring.

\subsection{Key Aspects}

The project focuses on several critical areas:
\begin{itemize}
    \item \textbf{Dynamic Route Adjustments:} The model adapts routes in real-time, rerouting to avoid areas impacted by the disaster.
    \item \textbf{Feasibility of Routes:} Ensures evacuation paths are viable, taking into account Montreal’s diverse urban terrain and avoiding areas directly affected by the disaster.
    \item \textbf{Comprehensive Coverage:} Aims to cover all strategic evacuation points across the city, ensuring no critical area is neglected in the evacuation plan.
\end{itemize}

\subsection{Practical Constraints}

The model is designed considering various operational constraints, including:
\begin{itemize}
    \item Minimizing Distance: Focuses on reducing the total distance traveled by evacuation buses, which is crucial for a quick and efficient evacuation.
    \item Vehicle Capacity: Maintains the safety and comfort of evacuees by adhering to the capacity limits of each bus.
    \item Efficient Utilization of Transit Network: Maximizes the use of available transit infrastructure to facilitate a smooth evacuation process.
\end{itemize}

\section{Methodology}
To address the challenges of disaster response in urban areas, this project proposes a model that optimizes evacuation routes for a fleet of buses.
In the academic literature, this is known as the \textit{Vehicle Routing Problem} (VRP), which is a generalization of the \textit{Traveling Salesman Problem} (TSP).
The VRP is a combinatorial optimization problem that seeks to identify the optimal set of routes for a fleet of vehicles tasked with servicing a set of demand nodes.


In this section, we outline the construction of the CVRP model aimed at optimizing disaster evacution routes.
The model seeks to identify the shortest possible routes for a fleet of buses tasked with evacuating residents to designated shelters, taking into account the constraints imposed by potential disaster scenarios that affect the city's transit network.

After the CVRP is formulated, we introduce a series of algorithms to convert the problem into a Split Delivery Vehicle Routing Problem (SD-VRP).
In a SD-VRP, a demand node may have capacity requirements that exceed the capacity of a single vehicle.
This necessitates formulating a problem that allows for that demand node to be visited by more than one vehicle.


\subsection{Data Collection and Preparation}

A comprehensive data collection forms the foundation of the Vehicle Routing Problem (VRP) model.
We utilized the General Transit Feed Specification (GTFS) from Société de transport de Montréal for detailed transit routes, schedules, and stop information.
After the requisite data was collected, a random sample of 20 bus stops was taken. Then, a user-defined depot can be added to the sample of stops.

Additionally, road distances between all nodes were obtained via the OSMnx package, providing information on the actual travel distances within the city's road network.
This integration of OSMnx data ensures that the model's distance calculations are approximately true to the real-world distances.
The collected data was cleaned and preprocessed to ensure compatibility with the modeling environment and to establish a reliable baseline for the optimization process.

\subsection{Mathematical formulation of CVRP}

The CVRP model is formulated as follows:
\subsubsection{Parameters}
Formally, the parameters to the model are defined as:
\begin{itemize}
    \item The set of stops is defined as $V = \{0, 1, 2, ..., n\}$, where $n$ is the number of stops. The depot is defined as the node ${0}$.
    \item The distance between two nodes $i$ and $j$ is denoted as $D_{ij}$ and is assumed to be symmetric, i.e., $D_{ij} = D_{ji}$.
    \item The demand at node $i, \forall i \in V - \{0\}$ is denoted as $q_i \sim U(L, H)$,
          where $L$ and $H$ are the lower and upper bounds of the uniform distribution, respectively.
    \item The capacity of each bus is denoted as $Q$
    \item The number of buses is denoted as $K$

\end{itemize}

\subsubsection{Decision Variables}
The decision variables to the model are defined as:
\begin{align*}
    x_{ijk} & = \begin{cases}
                    1 & \text{if the bus $k$ travels from node $i$ to node $j$} \\
                    0 & \text{otherwise}
                \end{cases} \\
    u_i     & = \text{the amount of demand satisfied at node $i$}
\end{align*}

\subsubsection{Formulation}
With the given parameters and decision variables, the model is given by:
\numberwithin{equation}{section}
\begin{alignat}{4}
     & \text{minimize:}   &       & \sum_{i=0}^{n} \sum_{j=0}^{n} \sum_{k=0}^{K} D_{ij} x_{ijk}                                                              \\
     & \text{subject to:} & \quad & \sum_{j=0}^{n} x_{ijk} = \sum_{j=0}^{n} x_{jik},            & \quad \forall i \in V, \forall k \in K                     \\
     &                    & \quad & \sum_{i=1}^{n} \sum_{k=0}^K x_{ijk} = 1,                    & \quad \forall j \in V - \{0\}                              \\
     &                    & \quad & \sum_{j=1}^{n} x_{0jk} \leq 1,                              & \quad \forall k \in K                                      \\
     &                    & \quad & \sum_{j=1}^{n} \sum_{i=0}^{n} q_j \cdot x_{ijk} \leq Q,     & \quad \forall k \in K                                      \\
     &                    & \quad & x_{iik} = 0,                                                & \quad \forall i \in V, \forall k \in K                     \\
     &                    & \quad & u_j - u_i \geq q_j - Q \cdot (1 - x_{ijk}),                 & \quad \forall i,j \in V - \{0\}, i \neq j, \forall k \in K \\
     &                    & \quad & q_i \leq u_i \leq Q,                                        & \quad \forall i \in V - \{0\}                              \\
     &                    & \quad & D_{ij} \cdot x_{ijk} \leq 5,                                & \quad \forall i,j \in V - \{0\}, \forall k \in K           \\
     &                    & \quad & x_{ijk} \in \{0,1\},                                        & \quad \forall i,j \in V, \forall k \in K                   \\
     &                    & \quad & u_i \in \mathbb{Z},                                         & \quad \forall i \in V
\end{alignat}

The objective function (3.1) minimizes the total distance traveled by all buses. Constraints (3.2) ensures that each bus leaves a node that it enters,
(3.3) ensures that each node is visited only once,
(3.4) ensures that each bus leaves the depot at most once,
(3.5) ensures that the capacity of each bus is not exceeded,
and (3.6) ensures that a bus does not travel from a node to itself.


Constraints (3.7) and (3.8) are the Miller-Tucker-Zemlin (MTZ) constraints to eliminate subtours,
i.e., cycling routes that do not pass through the depot.

Constraints (3.9) impose a distance restriction of 5 km on travel between any two non-depot nodes.
This is to ensure that the model does not generate routes that are infeasible in the real world.

Constraints (3.10) and (3.11) define the decision variables as binary and integer, respectively.

\end{document}